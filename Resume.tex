\documentclass[a4paper,11pt]{article}

%%%%%%% info %%%%%%%%%
\newcommand{\applicantname}{Tianyu Gao}
\newcommand{\applicantemail}{tygao@cmu.edu}


\usepackage{cvtemplate3}

%-----------------------------------------------------------
%\usepackage{latexsym}

\begin{document}

\thispagestyle{empty}

\cvname{\applicantname}
\cvaddress{Carnegie Mellon University}{5000 Forbes Ave, Pittsburgh, PA 15213}
\cvemail {Email:\  {\color{white} \applicantemail}\quad
Tel:\ 412-954-8158 }
\makecvtitle

\section{Education}
\cventry{Sept. 2015--\\Present}{Carnegie Mellon University}{Pittsburgh, PA}{}
{\begin{itemize}
\item Master of Science in Chemical Engineering
\item \textbf{Overall GPA: 3.95/4, Major GPA: 4/4}
\item Selected Courses: analysis and modeling of transport phenomenon, process systems modeling, principle and application of molecular simulation
\end{itemize}
}

\cventry{Sept. 2011--\\Jul. 2015}{Dalian University of Technology}{Dalian, China}{}
{\begin{itemize}
\item Bachelor of Engineering in Chemical Engineering and Technology
\item \textbf{Overall GPA: 90.4/100, Major GPA: 91.1/100}
\item Selected Courses: thermodynamics, unit operation, chemical reaction engineering
\end{itemize}
}


\section{Research \& Industrial Experience}
\cventry{Jan. 2016--\\Present}{Graduate Thesis} {Carnegie Mellon University, PA}{}{
\leftline{---Study of machine learned atomic metal potential energy surface} \rightline{Advisor: Prof. John Kitchin}
\begin{itemize}
\item Implemented density functional theory (DFT) and nudged elastic band (NEB) calculations using Vienna \textit{Ab initio} Simulation Package (VASP).
\item Applied a high dimensional neural networks (NN) method to model Pd potential energies surface and performed large time scale molecular dynamics (MD) simulations.
\item Achieved an excellent accuracy of modeling ground and transit state potential energies at a speed several order faster than DFT calculations.
\end{itemize}
}

\cventry{Sept. 2014--\\May 2015}{Undergraduate Thesis} {State Key Laboratory of Fine Chemicals, China}{}{
\leftline{---Study on coated bimetallic nanocatalyst preparation and application} \rightline{Advisor: Prof. Rongwen Lu}
\begin{itemize}
\item Prepared silica coated CuNi bimetallic nanoparticles from reverse microemulsion with uniform size and morphology.
%\item
\end{itemize}
}


\cventry{Apr. 2013--\\May. 2014} {Research Assistant}{State Key Laboratory of Fine Chemicals, China}{}{
\leftline{---Highly enhanced photocatalytic activity of Ag/AgCl/TiO$_{2}$ by CuO co-catalyst} \rightline{Advisor: Prof. Rongwen Lu}
\begin{itemize}
\item Synthesized TiO$_2$ coated Cu/Ag/AgCl nanoparticles in a reverse microemulsion system.
\item Evaluated photocatalytic activity by degradation of methyl orange and phenol under visible light.
\item Improved photocatalytic efficiency significantly and studied mechanism through band gap theory and surface plasma resonance.
\end{itemize}
}


\cventry{June 2014--\\July 2014} {Internship, Group Leader}{Shenyang Research Institute of Chemical Industry, China}{}{
\begin{itemize}
\item Simulated the process of propylene-propane distillation and designed affiliated facilities.
\item Experimented in a diazols dye synthesis and studied the process of industrialized scale up.
\end{itemize}
}


\section{Skills}

\cvitem {} { \textbf{Lab techniques:} Gas chromatography-mass spectrometry (GC-MS), high performance liquid chromatography (HPLC), ultraviolet-visible spectroscopy (UV-vis), transmission electron microscopy (TEM), X-ray powder diffraction (XRD)}
\cvitem {} {\textbf{Software:} VASP, Aspen Plus, Aspen Customer Model, GAMS, COMSOL Multiphysics, Simulink, Microsoft Office, ChemOffice, Origin}
\cvitem {} {\textbf{Programming Language:} Python, Matlab, C, \LaTeX}



\section{Publications}
\begin{itemize}
\item Yuzhen Ge, \textbf{Tianyu Gao}, Cui Wang, Rongwen Lu "Highly Efficient Silica Coated CuNi Bimetallic Nanocatalyst from Reverse Microemulsion", Journal of Colloid and Interface Science, In Press
\end{itemize}

\end{document}


