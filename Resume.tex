\documentclass[a4paper,11pt]{article}
%\documentclass[letterpaper,11pt]{article}

%%%%%%% info %%%%%%%%%
\newcommand{\applicantname}{Tianyu Gao}
\newcommand{\applicantemail}{gaotianyug94@gmail.com}


\usepackage{cvtemplate3}

%-----------------------------------------------------------
%\usepackage{latexsym}

\begin{document}

\thispagestyle{empty}

\cvname{\applicantname}
\cvaddress{Carnegie Mellon University}{515 S Aiken Ave, Apt 419, PA 15232}
\cvemail {Email:\  {\color{white} \applicantemail}\quad
Tel:\ 412-954-8158 }
\makecvtitle

\section{Education}
\cventry{Dec. 2016}{Carnegie Mellon University}{Pittsburgh, PA}{}
{\begin{itemize}
\item Master of Science in Chemical Engineering
\item \textbf{Overall GPA: 3.95/4, Major GPA: 4/4}
\item Selected Courses: Analysis and Modeling of Transport Phenomenon, Process Systems Modeling, Mathematical Modeling of Chemical Engineering Processes, Molecular Simulation
\end{itemize}
}

\cventry{Jul. 2015}{Dalian University of Technology}{Dalian, China}{}
{\begin{itemize}
\item B.S. with hornor in Chemical Engineering and Technology
\item \textbf{Overall GPA: 90.84/100, Major GPA: 91.80/100}
\item Selected Courses: Thermodynamics, Unit Operation, Chemical Reaction Engineering
\end{itemize}
}


\section{Research \& Industrial Experience}
\cventry{Jan. 2016--\\Present}{Graduate Thesis} {Carnegie Mellon University, PA}{}{
\leftline{---Study of machine learned atomic metal potential energy surface}
\begin{itemize}
\item Implemented density functional theory (DFT) and nudged elastic band (NEB) calculations using Vienna \textit{Ab initio} Simulation Package (VASP).
\item Applied a neural networks (NN) method to model Pd potential energies surface and performed large time scale molecular dynamics (MD) simulations for diffusion barrier estimation.
\item Achieved an excellent accuracy of modeling ground and transit state potential energies and energy barriers at a speed several order faster than DFT calculations.
\end{itemize}
}

%\cventry{Mar. 2016--\\May 2016}{Course Project} {Carnegie Mellon University, PA}{}{
%\leftline{---Optimization of profit for Aspirin manufacture process}
%\begin{itemize}
%\item Simulated Aspirin manufacture process in ASPEN and optimized profit using GAMS.
%\item Implemented a PID controller on the crystallizer to stabilize reactor temperature.
%\end{itemize}
%}

\cventry{Sept. 2014--\\May 2015}{Undergraduate Thesis} {State Key Laboratory of Fine Chemicals, China}{}{
\leftline{---Study on coated bimetallic nanocatalyst preparation and application}
\begin{itemize}
\item Prepared silica coated CuNi bimetallic nanoparticles from reverse microemulsion by modified co-reduction method and characterized particles composition, size and morphology.
\item Investigated catalysis activities of various compositions and sizes for \textit{p}-nitrophenol reduction.
\item Enhanced catalytic activity and selectivity compared to monometallic particles and studied bimetal synergetic effects.
\end{itemize}
}


\cventry{Apr. 2013--\\May 2014} {Research Assistant}{State Key Laboratory of Fine Chemicals, China}{}{
\leftline{---Highly enhanced photocatalytic activity of Ag/AgCl/TiO$_{2}$ by CuO co-catalyst}
\begin{itemize}
\item Synthesized TiO$_2$ coated Cu/Ag/AgCl nanoparticles in a reverse microemulsion system.
\item Evaluated photocatalytic activity by degradation of methyl orange and phenol under visible light.
\item Improved photocatalytic efficiency significantly and studied mechanism through band gap theory and surface plasma resonance.
\end{itemize}
}


\cventry{June 2014--\\July 2014} {Intern, Group Leader}{Shenyang Research Institute of Chemical Industry, China}{}{
\begin{itemize}
\item Simulated and optimized propylene-propane distillation process and designed affiliated facilities.
\item Experimented in a diazols dye synthesis and studied the process of industrialized scale up.
\end{itemize}
}


\section{Skills}

\cvarray{\textbf{Lab techniques:} Gas chromatography-mass spectrometry (GC-MS), high performance liquid chromatography (HPLC), ultraviolet-visible spectroscopy (UV-vis), transmission electron microscopy (TEM), Fourier transform infrared spectroscopy (FT-IR), X-ray diffraction (XRD)}
\cvarray{\textbf{Software:} VASP, Aspen Plus, Aspen Customer Model, GAMS, COMSOL Multiphysics, Simulink, Microsoft Office, ChemOffice, Origin}
\cvarray{\textbf{Programming Language:} Python, Matlab, C, \LaTeX}



\section{Publications}
\cvarray {Yuzhen Ge, \textbf{Tianyu Gao}, Cui Wang, Rongwen Lu, "Highly Efficient Silica Coated CuNi Bimetallic Nanocatalyst from Reverse Microemulsion," \textit{Journal of Colloid and Interface Science}, under third review.}
\cvarray {\textbf{Tianyu Gao}, John Kitchin, "Modeling Palladium surfaces with Density Functional Theory and Neural Networks," submitted to \textit{Surface Science}}

\end{document}


